\documentclass[a4paper,12pt]{article}

\include{preamble}

\title{Travaux dirigés 1 - Phénomènes de diffusion}
\author{L2 CUPGE PHYSIQUE}
\date{Semestre 4}

\begin{document}
\maketitle

\section{Isolation thermique des maisons}

\subsection{Question de cours}

On rappelle l'expression de la loi élémentaire de Fourier donnant le vecteur densité de courant de chaleur $\vv{J_Q}$.
\begin{equation}
	\vv{J_Q} = - \kappa \vv{\nabla} T(M,t)
\end{equation}
Dans notre situation on peut se ramener à une étude à une dimension, la température ne dépend alors que de $x$ et $t$ : $T(x,t)$. On obtient donc la relation suivante :
\begin{equation}
	\vv{J_Q} = - \kappa \frac{\partial T(x,t)}{\partial x} \vv{u_x}
\end{equation}

On réalise ensuite un bilan des échanges de chaleur pendant un intervalle de temps $\diff{t}$ sur un petit volume de section $\diff{S}$ et d'épaisseur $\diff{x}$.
\begin{equation}
	\begin{split}
		\delta Q &= \delta Q_\text{entrant} - \delta Q_\text{sortant}\\
		&= J_Q ( x+\diff{x},t ) \diff{S}\diff{t} - J_Q (x,t) \diff{S}\diff{t}\\
		&= \left[ J_Q (x+\diff{x},t) - J_Q (x,t) \right] \diff{S}\diff{t}\\
		\delta Q &= \frac{\partial J_Q (x,t)}{\partial x} \diff{x} \diff{S} \diff{t}
	\end{split}
\end{equation}

On cherche alors à monter que la température $T(x,t)$ vérifie l"quation de Fourier $\frac{\partial T(x,t)}{\partial t} = D \frac{\partial^2 T(x,t)}{\partial x^2}$. On utilise pour cela la chaleur spécifique par unité de volume $C$. On peut l'exprimer comme :
\begin{equation}
	\begin{split}
		C &= \frac{C_v}{V}\\
		  &= \frac{1}{V} \cdot \frac{\diff{U}}{\diff{T}}\\
		C &= \frac{1}{V} \cdot \frac{\delta Q}{\diff{T}} \text{ car dans notre cas } \delta W = 0
	\end{split}
\end{equation}

Ainsi on peut exprimer $\delta Q$ avec cette nouvelle expression. On obtient :
\begin{equation}
	\delta Q = C \diff{T} \diff{x}\diff{S}
\end{equation}
On obtient donc l'égalité suivante :
\begin{equation}
	\delta Q = C \diff{T}\diff{x}\diff{S} = \frac{\partial J_Q (x,t)}{\partial x} \diff{x} \diff{S} \diff{t}
\end{equation}
Or d'après la relation de Fourier, ici à une dimension, on sait que 
\begin{equation}
	\vv{J_Q} = - \kappa \frac{\partial T(x,t)}{\partial x} \vv{u_x}
\end{equation}
En réinjectant ce résultat dans l'expression ci-dessus et en le simplifiant on obtient alors :
\begin{equation}
	\frac{\diff{T}}{\diff{t}} = -\frac{\kappa}{C} \frac{\partial^2 T}{\partial x^2}
\end{equation}
En posant $D = -\frac{\kappa}{C}$ on obtient le résultat recherché.

Que devient cette équation en régime permanent ? En régime permanent le déséquilibre est créé en continu ainsi on a
\begin{equation}
	\frac{\partial^2 T}{\partial x^2} = 0
\end{equation}

\subsection{Le mur séprare un milieu à la température $T_\text{int}$ (pour $x < 0$) d'un autre milieu à l'intérieur à la température $T_\text{ext}$ (pour $x < e$).}

On détermine, en régime permanent, la répartition de température $T(x)$ à l'intérieur du mur. On suppose $T_\text{int}$ et $T_\text{ext}$ constant, donc que le système est en déséquilibre de façon permanente. On a alors 
\begin{equation}
	\frac{\partial^2 T}{\partial x^2} = 0 \Rightarrow T(x) = Ax + B
\end{equation}
Sachant $T(0) = T_\text{int}$ on a $B = T_\text{int}$. Ensuite on a $T(e) = T_\text{ext}$ donc on peut calculer $A$ :
\begin{equation}
	A = \frac{T_\text{ext} - T_\text{int}}{e}
\end{equation}
L'expression de $T(x)$ devient alors
\begin{equation}
	T(x) = \frac{T_\text{ext} - T_\text{int}}{e} x + T_\text{int}
\end{equation}
On utilise finalement la loi élémentaire de Fourier :
\begin{equation}
	J_Q (x) = - \kappa \frac{\partial T(x)}{\partial x} = - \kappa \frac{T_\text{ext} - T_\text{int}}{e}
\end{equation}

On peut ensuite calculer la puissance qu'il faut fournir au milieu à la température $T$ la plus élevée afin de compenser les pertes de chaleur à travers le mur. Pour cela on rappelle que la puissance est une énergie par unité de temps :
\begin{equation}
	P = \frac{\diff{U}}{\diff{t}} = \frac{\delta Q(t)}{\diff{t}} = J_Q (x,t) \cdot S = - \kappa \frac{\frac{T_\text{ext} - T_\text{int}}{e} \cdot S \diff{t}}{\diff{t}}
\end{equation}
L'expression de la puissance est donc simplement
\begin{equation}
	P = - \kappa \frac{T_\text{ext} - T_\text{int}}{e} \cdot S
\end{equation}
\emph{Application numérique :} $P = \SI{1875}{\watt}$.

On peut montrer que les formules obtenues peuvent être comparées à celles que l'on obtiendrait en électricité avec le champ électrique $\vv{E}$, le champ de potentiel $\vv{\nabla} V$ et la conductivité électrique $\sigma$ et ainsi définir une \og résistance thermique \fg{} du mur.

En électricité on connait la très célèbre relatio $U=RI$. On définit $I$ comme
\begin{equation}
	I = \int_S \vv{J} . \vv{n} \diff{S} = J S
\end{equation}
avec
\begin{equation}
	J = - \sigma \vv{E} = \sigma \vv{\nabla} V
\end{equation}
On a ici une analogie entre $J_Q$ et $J$, avec $\sigma$ la conductivité électrique, $\kappa$ la conductivité thermique, $\vv{E}$ ou $\vv{- \nabla} V$ le champ de potentiel. Ainsi la loi d'ohm peut être utilisé par analogie pour définir $R_\text{th}$ la résistance thermique. La différence de potentiel analogue à $U$ dans notre cas peut se noter sous la forme d'une différence de température $\Delta T$. Ainsi on obtient :
\begin{equation}
	R_\text{th} = \frac{\Delta T}{I_Q} = \frac{\Delta T}{J_Q S} = \frac{\Delta T}{\kappa \frac{\Delta T}{e} S} = \frac{e}{\kappa S}
\end{equation}
On vérifie la dimension de notre expression. On devrait trouver la dimension
\begin{equation}
	[R] = \mathrm{K} [P]^{-1} = \mathrm{K} \mathrm{M}^{-1} \mathrm{L}^{-2} \mathrm{T}^{3}
\end{equation}
\begin{equation}
	[R] 	= \frac{[e]}{[\kappa][S]} 
		= \frac{\mathrm{L}}{\mathrm{M}\mathrm{L}^1\mathrm{T}^{-3}\mathrm{K}^{-1}  \mathrm{L}^2} 
		= \frac{\mathrm{L}}{\mathrm{M}\mathrm{L}^3\mathrm{T}^{-3}\mathrm{K}^{-1}}
		= \mathrm{K}\mathrm{M}^{-1}\mathrm{L}^{-2}\mathrm{T}^{3}
\end{equation}
\emph{Application numérique : } $R_\text{th} = \SI{8e-3}{\kelvin\per\watt}$.

\subsection{Application au double vitrage}

On détermine maintenant la valeur numérique de la résistance thermique pour une plaque de verre de surface ${S = \SI{1}{\metre\squared}}$ d'épaisseur ${e=\SI{4}{\milli\metre}}$ et de conductivité thermique ${\kappa = \SI{0.8}{\watt\per\metre\per\kelvin}}$.

\noindent \emph{Application numérique : } $R_\text{th} = \SI{5e-3}{\kelvin\per\watt}$.

Si l'air extérieur est porté à la température de ${T_\text{ext} = \SI{20}{\celsius}}$ et s'il règne une température ${T_\text{int} = \SI{0}{\celsius}}$ à l'extérieur, on peut évaluer la puissance thermique perdu par cette même vitre. Celle-ci correspond à la chaleur qui traverse la section $S$ de la vitre par unité de temps. On a
\begin{equation}
	P = J_Q S = -\kappa{\Delta T}{e} \cdot S
\end{equation}
On remarque alors que l'on peut réécrire cette équation avec le terme de résistance thermique et on obtient alors :
\begin{equation}
	P = \frac{\Delta T}{R_\text{th}}
\end{equation}
\emph{Application numérique : } $P_\text{perdue} = \SI{4}{\kilo\watt}$.

On peut ensuite déterminer la valeur numérique de la résistance thermique d'un double vitrage constitué de deux vitres identiques à celle définie précedemment mais d'épaisseur ${e_2 = \SI{2}{\milli\metre}}$ chacune et d'une épaisseur $e_2$ d'air sec de conductivité thermique ${\kappa_2 = \SI{2.6e-2}{\watt\per\metre\per\kelvin}}$. Pour cela, toujours par analogie avec l'électrocinétique on, on additionne les résistances, comme s'il s'agissait de résistances électriques en séries. On obtient alors une résistance équivalentes $R_\text{éq}$
\begin{equation}
	\begin{split}
		R_\text{éq} 	&= 2R_\text{vitre} + R_\text{air}\\
				&= 2 \left( \frac{e_2}{S\kappa} \right) + \frac{e_2}{S\kappa_2}\\
		R_\text{éq}	&= \frac{2e_2}{S\kappa} + \frac{e_2}{S\kappa_2}
	\end{split}
\end{equation}
\emph{Application numérique : } $R_\text{éq} = \SI{8e-2}{\kelvin\per\watt}$. On peut alors facilement en déduire la puissance thermique perdue : ${P_\text{perdue} = \SI{244}{\watt}}$. On en conclut donc qu'il est plus utile de réduire la conductivité thermique que d'augmenter l'épaisseur. Ainsi le double vitrage est nettement plus effectif (environ 20 fois) qu'une vitre de la même épaisseur.

\section{Diffusion de neutrons dans un barreau d'Uranium}


\end{document}
